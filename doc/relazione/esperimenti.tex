\chapter{Esperimenti}

\section{Cluster ChemGrid}

Tutti gli esperimenti sono stati eseguiti sul cluster ChemGrid. Il cluster \`{e} composto da 11 nodi a 32 bit con 2 processori Intel(R) Pentium(R) 4 CPU 3.06GHz. Tutti i nodi hanno 1GB di memoria tranne 4 nodi (8, 9 10, 11) che hanno 4GB. 3 nodi (3, 6, 11) sono offline e dunque non utilizzabili per sottomettere job. Dunque i nodi utilizzabili sono 8 per un totale di 16 CPU.

Da ricordare che ci serve un quadrato perfetto per eseguire l'algoritmo di Cannon: 4, 9 e 16 sono i possibili candidati. 9 non \`{e} possibile utilizzarlo perch \`{e} non si hanno a disposizione sufficienti host.
Il 4 potrebbe essere derivato da 4 host con una sola CPU oppure da 2 host con 2 CPU. Infine 16 \`{e} ottenibile solo con 8 host a 2 CPU l'uno.
Tutti gli esperimenti verranno eseguiti con questa ultima configurazione: 8 host e 2 CPU per nodo, per un totale di 16 CPU.

\section{Data file}

Il file di input utilizzato per gli esperimenti ha le seguenti dimensioni:

\begin{lstlisting}
4 4 4 2
8 8 8 2
16 16 16 2
32 32 32 2
64 64 64 2
128 128 128 2
256 256 256 2
512 512 512 2
1024 1024 1024 2
2048 2048 2048 2
4096 4096 4096 2
8192 8192 8192 2
16384 16384 16384 2
\end{lstlisting}

Non ha senso andare oltre 16384 perch\`{e} i nodi iniziano ad utilizzare la memoria swap. Si consideri che una matrice quadrata con n = 16384 ha 268435456 elementi. Ogni elemento \`{e} un double che occupa 8 bytes, dunque una matrice occupa circa 2GB.
Tutti i test sono stati fatti utilizzando 2 iterazioni.

\subsection{Seriale}

La versione seriale \`{e} stata eseguita su un nodo (cg11) che ha 4GB di memoria per permettere alle dimensioni pi\`{u} grandi si essere moltiplicate senza problemi. Di seguito il grafico delle sue prestazioni.

\subsection{MPI senza ottimizzazioni}
\paragraph{Speedup ed efficienza}

\subsection{MPI non bloccante}
\paragraph{Speedup ed efficienza}

\subsection{MPI con OpenMP}

\subsection{MPI con CBLAS}
